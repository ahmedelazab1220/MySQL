
1. What is the purpose of the FLUSH PRIVILEGES command in MySQL, and when would you use it?

The FLUSH PRIVILEGES command is used to reload the in-memory privileges from the privilege tables in the MySQL database.
You would use this command when you manually modify the privilege tables and want to apply these changes without restarting the MySQL server.
#############################################################################################################################################################################################

2. Explain the difference between TRUNCATE and DELETE commands. Provide at least two key differences.

- DELETE removes rows from a table but allows you to filter which rows to delete using a WHERE clause. 
  On the other hand, TRUNCATE removes all rows from a table without using a WHERE clause.

- TRUNCATE is faster and uses less transaction log space compared to DELETE because it deallocates the data pages. 
  Also, TRUNCATE resets any auto-increment counters, while DELETE does not.

#############################################################################################################################################################################################

3. Describe the difference between VARCHAR and CHAR data types.

- CHAR is a fixed-length data type. If you define a CHAR(10) and insert a string of fewer than 10 characters, MySQL will pad the string with spaces to make it exactly 10 characters long.

- VARCHAR is a variable-length data type. If you define a VARCHAR(10) and insert a string of 5 characters, only those 5 characters are stored, and no extra padding is added.

#############################################################################################################################################################################################

4. What is the purpose of the REVOKE command? Provide an example of revoking INSERT and UPDATE privileges on a specific table.

- The REVOKE command is used to remove previously granted privileges from a user. 
  For example, to revoke INSERT and UPDATE privileges on the employees table from user emp_user, you would write:

  REVOKE INSERT, UPDATE ON company_db.employees FROM 'emp_user'@'localhost';

#############################################################################################################################################################################################

5. In MySQL, what is the difference between a PRIMARY KEY and a UNIQUE KEY? Can a table have more than one PRIMARY KEY

- A PRIMARY KEY uniquely identifies each row in a table and cannot contain NULL values. A table can only have one PRIMARY KEY.

- A UNIQUE KEY also ensures uniqueness for a column, but it can contain a single NULL value, and a table can have multiple UNIQUE KEY columns.

- No, a table cannot have more than one PRIMARY KEY, but it can have multiple UNIQUE constraints.

#############################################################################################################################################################################################


